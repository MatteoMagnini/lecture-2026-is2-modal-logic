%! Author = matteomagnini
%! Date = 17/02/26

% Preamble
\documentclass[11pt]{article}

% Packages
\usepackage{amsmath}
\usepackage{amssymb}

\title{
    Intelligent System 2\\
    Tutorial week 1: Propositional logic
}

\date{
    20 February, 2026
}


% Document
\begin{document}

    \maketitle

    \section{Exercises}
    \label{sec:exercises}

    \subsection*{Exercise 1}

    Give the subformulas of the following propositional formulae:

    \begin{enumerate}
        \item $(p_1 \vee p_2) \wedge (p_3 \vee p_4)$
        \item $\neg (\neg p_1 \wedge \neg p_2)$
        \item $(\neg p_1 \wedge (\neg p_2 \rightarrow p_3))$
    \end{enumerate}

    \subsection*{Solution to Exercise 1}

    Below are all the subformulas for each given propositional formula:

    \begin{enumerate}
        \item \textbf{$(p_1 \vee p_2) \wedge (p_3 \vee p_4)$}

        The subformulas are:
        \begin{itemize}
            \item $(p_1 \vee p_2) \wedge (p_3 \vee p_4)$
            \item $(p_1 \vee p_2)$
            \item $p_1$
            \item $p_2$
            \item $(p_3 \vee p_4)$
            \item $p_3$
            \item $p_4$
        \end{itemize}

        \item \textbf{$\neg (\neg p_1 \wedge \neg p_2)$}

        The subformulas are:
        \begin{itemize}
            \item $\neg(\neg p_1 \wedge \neg p_2)$
            \item $(\neg p_1 \wedge \neg p_2)$
            \item $\neg p_1$
            \item $p_1$
            \item $\neg p_2$
            \item $p_2$
        \end{itemize}

        \item \textbf{$(\neg p_1 \wedge (\neg p_2 \rightarrow p_3))$}

        The subformulas are:
        \begin{itemize}
            \item $(\neg p_1 \wedge (\neg p_2 \rightarrow p_3))$
            \item $\neg p_1$
            \item $p_1$
            \item $(\neg p_2 \rightarrow p_3)$
            \item $\neg p_2$
            \item $p_2$
            \item $p_3$
        \end{itemize}
    \end{enumerate}

    
    \subsection*{Exercise 2}

    Soundness and completeness are important properties of proof calculi.
    Why is it important that both properties are met at the same time?
    To illustrate your answer, please give a definition of a provability predicate $\vdash$ (i.e.\ ``$\vdash A$'' means $A$ is provable) that\ldots

    \begin{enumerate}
        \item is sound for propositional logic, but not complete.
        \item is complete for propositional logic, but not sound.
    \end{enumerate}

    \subsection*{Solution to Exercise 2}

    \textbf{Answer:}

    Soundness ensures that every provable formula is also a semantic truth (valid in all interpretations),
    while completeness guarantees that every semantic truth is provable in the calculus.
    %
    Both properties are essential:
    if only soundness holds,
    the calculus might fail to prove some true formulas (it is too weak);
    if only completeness holds,
    the calculus may allow the proof of formulas that are not actually true (it is too permissive).
    %
    Only when both soundness and completeness hold can we be confident that ``provable'' and ``true'' coincide.

    \begin{enumerate}
        \item \textbf{A provability predicate that is sound but not complete:}

        Define $\vdash A$ to mean ``$A$ is of the form $p \vee \neg p$ for some variable $p$\,'', i.e., only certain tautologies are considered provable, while most valid formulas are not.

        Formally,
        $$
        \vdash A \iff A \text{ is syntactically of the form } (p \vee \neg p)
        $$
        This system is sound (it never proves something false), but it is not complete (there are many valid formulas that cannot be proved).

        \item \textbf{A provability predicate that is complete but not sound:}

        Define $\vdash A$ to mean ``$A$ is any propositional formula'', i.e., every formula is provable, regardless of whether it is valid.

        Formally,
        $$
        \vdash A \quad \text{for every formula } A
        $$
        In this case, every valid formula is provable (completeness), but also invalid ones are provable (not sound).
    \end{enumerate}

    
    \subsection*{Exercise 3}

    Provide all variable assignments verifying the following propositional formulae:

    \begin{enumerate}
        \item $\neg p_1 \vee p_2$
        \item $\neg p_1 \wedge (p_2 \rightarrow p_3)$
        \item $(p_1 \rightarrow p_2) \wedge (\neg p_3)$
    \end{enumerate}

    \subsection*{Solution to Exercise 3}

    For each formula, list all assignments of propositional variables that make the formula true.

    \begin{enumerate}
        \item $\neg p_1 \vee p_2$

        The formula is true in all cases except when $p_1 = \text{True}$ and $p_2 = \text{False}$.
        \begin{itemize}
            \item $p_1 = \text{False}$, $p_2 = \text{False}$: True
            \item $p_1 = \text{False}$, $p_2 = \text{True}$: True
            \item $p_1 = \text{True}$, $p_2 = \text{True}$: True
            \item $p_1 = \text{True}$, $p_2 = \text{False}$: False
        \end{itemize}

        \item $\neg p_1 \wedge (p_2 \rightarrow p_3)$

        The formula is true if $p_1 = \text{False}$ and either $p_2 = \text{False}$ or $p_3 = \text{True}$.
        \begin{itemize}
            \item $p_1 = \text{False}$, $p_2 = \text{False}$, $p_3 = \text{False}$: True
            \item $p_1 = \text{False}$, $p_2 = \text{False}$, $p_3 = \text{True}$: True
            \item $p_1 = \text{False}$, $p_2 = \text{True}$, $p_3 = \text{True}$: True
            \item $p_1 = \text{False}$, $p_2 = \text{True}$, $p_3 = \text{False}$: False
            \item Any assignment with $p_1 = \text{True}$: False
        \end{itemize}

        \item $(p_1 \rightarrow p_2) \wedge (\neg p_3)$

        The formula is true if $p_3 = \text{False}$ and either $p_1 = \text{False}$ or $p_2 = \text{True}$.
        \begin{itemize}
            \item $p_1 = \text{False}$, $p_2 = \text{False}$, $p_3 = \text{False}$: True
            \item $p_1 = \text{False}$, $p_2 = \text{True}$, $p_3 = \text{False}$: True
            \item $p_1 = \text{True}$, $p_2 = \text{True}$, $p_3 = \text{False}$: True
            \item $p_1 = \text{True}$, $p_2 = \text{False}$, $p_3 = \text{False}$: False
            \item Any assignment with $p_3 = \text{True}$: False
        \end{itemize}
    \end{enumerate}


    \subsection*{Exercise 4}

    Simplify the following propositional formulas step-by-step by repeatedly applying De Morgan's Laws and other logical equivalences:

    \begin{enumerate}
        \item $\neg ((p \vee q) \wedge (\neg r \vee s))$

        \item $\neg (\neg(p \wedge q) \vee (\neg r \wedge t))$

        \item $\neg ((\neg(p \vee q)) \wedge \neg(\neg r \vee s))$

        \item $\neg ((\neg p \vee q) \wedge (\neg t \vee \neg u))$
    \end{enumerate}

    \subsection*{Solution to Exercise 4}

    \begin{enumerate}
        \item
        $\neg \big[ (p \vee q) \wedge (\neg r \vee s) \big]$
        \begin{align*}
            &= \neg (p \vee q) \vee \neg (\neg r \vee s) \quad\text{(De Morgan's Law)} \\
            &= (\neg p \wedge \neg q) \vee (\neg (\neg r) \wedge \neg s) \quad\text{(De Morgan's Law)} \\
            &= (\neg p \wedge \neg q) \vee (r \wedge \neg s) \quad\text{(Double negation)}
        \end{align*}

        \item
        $\neg \big[ \neg(p \wedge q) \vee (\neg r \wedge t) \big]$
        \begin{align*}
            &= \neg \neg(p \wedge q) \wedge \neg (\neg r \wedge t) \quad\text{(De Morgan's Law)} \\
            &= (p \wedge q) \wedge (\neg (\neg r) \vee \neg t) \quad\text{(De Morgan's Law)} \\
            &= (p \wedge q) \wedge (r \vee \neg t) \quad\text{(Double negation)}
        \end{align*}

        \item
        $\neg \big[ \neg(p \vee q) \wedge \neg(\neg r \vee s) \big]$
        \begin{align*}
            &= \neg \neg(p \vee q) \vee \neg \neg(\neg r \vee s) \quad\text{(De Morgan's Law)} \\
            &= (p \vee q) \vee (\neg \neg(\neg r \vee s)) \quad\text{(Double negation)} \\
            &= (p \vee q) \vee (\neg r \vee s) \quad\text{(Double negation)}
        \end{align*}

        \item
        $\neg \big[ (\neg p \vee q) \wedge (\neg t \vee \neg u) \big]$
        \begin{align*}
            &= \neg (\neg p \vee q) \vee \neg (\neg t \vee \neg u) \quad\text{(De Morgan's Law)} \\
            &= (p \wedge \neg q) \vee (t \wedge u) \quad\text{(De Morgan's Law)}
        \end{align*}
    \end{enumerate}

    \subsection*{Exercise 5}

    Show the following propositional formulae are not valid.
    If they are satisfiable, provide a variable assignment verifying them:

    \begin{enumerate}
        \item $(p_1 \vee p_2) \wedge (p_3 \vee p_4)$
        \item $\neg (\neg p_1 \wedge \neg p_2)$
        \item $(\neg p_1 \wedge (\neg p_2 \rightarrow p_3))$
    \end{enumerate}

    \subsection*{Solution to Exercise 5}

    \begin{enumerate}
        \item \textbf{$(p_1 \vee p_2) \wedge (p_3 \vee p_4)$}

        This formula is not valid, as it is not true under all assignments.
        For example, if $p_1 = \text{False}$, $p_2 = \text{False}$, $p_3 = \text{False}$, $p_4 = \text{False}$, the formula evaluates to $\text{False}$.

        However, it is satisfiable. For instance, if $p_1 = \text{True}$, $p_2 = \text{False}$, $p_3 = \text{True}$, $p_4 = \text{False}$, the formula evaluates to $\text{True}$.

        \item \textbf{$\neg (\neg p_1 \wedge \neg p_2)$}

        This formula is not valid, since for $p_1 = \text{False}$ and $p_2 = \text{False}$, $\neg p_1 = \text{True}$, $\neg p_2 = \text{True}$, and the formula becomes $\neg (\text{True} \wedge \text{True}) = \neg \text{True} = \text{False}$.

        It is satisfiable, for example, with $p_1 = \text{True}$, $p_2 = \text{False}$ (the formula is $\neg (\text{False} \wedge \text{True}) = \neg \text{False} = \text{True}$).

        \item \textbf{$(\neg p_1 \wedge (\neg p_2 \rightarrow p_3))$}

        This formula is not valid. For example, if $p_1 = \text{True}$, $p_2 = \text{True}$, $p_3 = \text{False}$, then $\neg p_1 = \text{False}$ and the whole formula is $\text{False}$.

        It is satisfiable. For example, if $p_1 = \text{False}$, $p_2 = \text{True}$, $p_3 = \text{False}$:
        \begin{itemize}
            \item $\neg p_1 = \text{True}$
            \item $\neg p_2 = \text{False}$
            \item $(\neg p_2 \rightarrow p_3) = (\text{False} \rightarrow \text{False}) = \text{True}$
            \item Therefore, $\text{True} \wedge \text{True} = \text{True}$
        \end{itemize}
    \end{enumerate}


\end{document}